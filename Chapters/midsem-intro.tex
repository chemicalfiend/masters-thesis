\chapter{Introduction} % Main chapter title

\label{Chapter1} % Change X to a consecutive number; for referencing this chapter elsewhere, use \ref{ChapterX}

\lhead{Chapter 1. \emph{Introduction}} % Change X to a consecutive number; this is for the header on each page - perhaps a shortened title

The electronic properties of solids are often dictated by how the electrons in the solid interact with each other, with the lattice and with light. Studying how these interactions work is important, because it helps us determine whether a material will work well as a conductor, a battery or a solar cell. An electron in a solid creates a local distortion of the lattice, pulling positive ions closer to it and pushing negative ions further away from it. When this electron is mobile and moves around, we expect that this local distortion also moves around. An electron and the lattice distortion around it are treated together as a single entity or quasiparticle called a "polaron". We often think of this moving distortion as a phonon, so we model this system as interacting electrons and phonons.

We know how to solve the electronic problem in solids and molecules, i.e. we can obtain the quantum states for electrons in an electrostatic potential of atoms in the solid using methods such as Density Functional Theory, CASSCF, etc. We also know how to get the phonon modes in a solid.

To better understand organic semiconductor materials, we want to understand how to derive the time-evolution of coupled electrons and phonons. To do this, we split the model into a "system" and a "bath"/"reservoir", where the system is the electron and the phonon modes form the bath that is coupled to the electron. We develop a rigorous formalism to separate the model into a system and an interacting bath and we use the path integral formalism to obtain a solvable equation, which we compute the solutions of using a variety of monte carlo calculations. We apply our methodology to simplified models of molecules such as Rubrene and Pentacene
