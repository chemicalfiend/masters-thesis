% Chapter Template

\chapter{Organic Semiconductors} % Main chapter title

\label{Chapter2} % Change X to a consecutive number; for referencing this chapter elsewhere, use \ref{ChapterX}

\lhead{Chapter 2. \emph{Organic Semiconductors}} % Change X to a consecutive number; this is for the header on each page - perhaps a shortened titl

\section{Introduction}

Organic molecular solids have attracted attention for a variety of applications, as structural materials, in medicine and as semiconductors, for classical transistor applications as OFETs (Organic Field Effect Transistors), or in optoelectronics as OLED (Organic Light Emitting Diode) display materials and OPV (Organic Photovoltaics) solar cells.

One of the issues with photovoltaic materials is the Shockley-Queisser limit, which is a theoretical limit for the efficiency of a solar cell. We exploit interesting properties of organic materials to overcome this limit to get highly efficient solar cells, specifically singlet fission and upconversion. Materials such as Y6 have showcased impressive efficiencies near 20\%  [cite].

\section{Exciton Generation}

Excitons are generated when an electron gets excited and leaves behind a hole. In photovoltaics, we are interested in charge separation, so we want to % Explain further.

Excited singlet states are formed when an electron gets excited such that the overall spin quantum number s = 0 (the multiplicity hence being 2s + 1 = 1)

% Diagram for singlet state in a dimer. 

Excited triplet states are formed when an electron gets excited such that the overall spin quantum number s = 1 (the multiplicity hence being 2s + 1 = 3). 

% Diagram for triplet state in a dimer.

For the example a dimer, the excited state can also form a charge transfer (CT) state.



\section{Singlet Fission}

Singlet fission is a photophysical process that may allow us to improve photoconversion efficiencies in solar cells. This is a process whereby one singlet excited state forms two triplet excited states. \cite{smith2010singlet}.


\section{Upconversion}


\section{Pentacene}

\section{Rubrene}

\section{Y6}

\section{Charge Transport}

\cite{oberhofer2017charge}


\subsection{Band Regime}

\subsection{Hopping Regime}

\subsection{Intermediate Regime}