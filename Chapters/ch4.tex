% Chapter Template

\chapter{Path Integrals and Quantum Dynamics} % Main chapter title

\label{Chapter4} % Change X to a consecutive number; for referencing this chapter elsewhere, use \ref{ChapterX}

\lhead{Chapter 4. \emph{Path Integrals and Quantum Dynamics}} % Change X to a consecutive number; this is for the header on each page - perhaps a shortened titl


\section{Introduction}

If we want to understand the electronic properties of materials, our limited understanding of analytically solvable quantum systems does not get us very far. There's only a limited number of systems with analytical solutions.

As in the case of classical mechanics, any realistic depiction of quantum systems would require understanding how the system couples with an environment, which influences it heavily. Open quantum system theory hence helps us with this problem since we can learn what happens to systems that interact with an environment and how that affects their dynamics. The easiest models of open quantum systems are called "system-bath" models where you have a (reasonably) solvable system coupled to a bath that represents the effects of an environment.

For studying electron-phonon systems, we need to construct models whereby we can understand what happens to the electrons when they interact with an ionic lattice as an environment. This chapter starts by covering the general theory of open quantum systems and the best ways to treat their evolution. We then move on to the path integral formulation and discuss how that helps us with the quantum dynamics of system-bath models. Once we have constructed these path integral equations, we move onto evaluating the path integrals using Monte Carlo simulations. 


\section{Dynamics of Open Quantum Systems}

For isolated quantum systems, the dynamics can be described by the time-dependent Schr{\"o}dinger equation (TDSE): \begin{equation}\label{4.1} -i\hbar \frac{\partial}{\partial t} \ket{\Psi(x, t)} = \mathcal{H} \ket{\Psi(x, t)}\end{equation}

The average values of a time-dependent observable A is given by \begin{equation} \label{4.2} A(t) = \bra{\Psi(t)} \hat{A} \ket{\Psi(t)} \end{equation}

One of our objectives is to figure out how a quantum-mechanical system moves from an initial state $t_0$ to a final state. A convenient tool to describe the states of macroscopic systems that can be formed from a mixture of pure states $\{ \Psi_{\nu} \}$ is the density operator $\hat{W}$, given by \begin{equation} \label{4.3} \hat{W} = \sum_{\nu} w_{\nu} \ket{\Psi_{\nu}} \bra{\Psi_{\nu}} \end{equation}which is a summation over the projection operators onto the pure states with a probability weight of $w_{\nu}$. Expectations for observables can now be determined by : $<\hat{A}> = Tr\{ \hat{W} \hat{A}\}$.

First let us try to construct an equation of motion for the density operator. 

\subsection{Liouville-von Neumann Equation}

John von Neumann extended the classical Lioville equation for phase space trajectories to quantum statistics.

The time propagation of the wavefunctions in eq. \ref{4.1} can be written as \begin{equation} \label{4.4} \ket{\Psi(t)} = U(t, t_0) \ket{\Psi(t_0)} \end{equation} where $$U(t, t_0) = \exp(-\frac{i \mathcal{H}(t - t_0)}{\hbar})$$.

We can substitute eq \ref{4.4} in in eq \ref{4.3} to obtain a time-dependent equation for the density operator.

\begin{equation} \label{4.5}
    \hat{W}(t) = \sum_{\nu} w_{\nu} U(t, t_0) \ket{\Psi(t_0)} \bra{\Psi(t_0)} U^{\dag}(t, t_0) = \sum_{\nu} U(t, t_0) W(t_0) U^{\dag}(t, t_0)
\end{equation}

To derive a time-derivative equation, take a partial derivative on both sides with respect to t:

\begin{equation} \label{4.6}
    \frac{\partial}{\partial t} \hat{W}(t) = -\frac{i}{\hbar}(\mathcal{H} \hat{W} - \hat{W} \mathcal{H}) = -\frac{i}{\hbar}[\mathcal{H}, \hat{W}(t)]
\end{equation}

Equation \ref{4.6} is called the Liouville-von Neumann equation and is the simplest equation for the evolution of macroscopic quantum systems. For brevity we can use the notation of super-operator $\mathcal{L} = \frac{1}{\hbar}[H, (.)]$ to get $$\frac{\partial}{\partial t} \hat{W} = -i \mathcal{L} \hat{W}$$

\subsection{System-Bath Models and The Reduced Density Matrix}

For models where we have a system coupled to a bath, we will represent our Hamiltonian in the form : 

\begin{equation} \label{4.7}
    \mathcal{H} = \mathcal{H}_S + \mathcal{H}_R + \mathcal{H}_{int}
\end{equation}

Where $\mathcal{H}_S$ is the Hamiltonian of the relevant system, represented only in terms of system coordinates $s = \{s_j\}$ and momenta $p = \{p_j\}$, $\mathcal{H}_R$ is the Hamiltonian of the reservoir having coordinates $Z = \{Z_{\xi}\}$ and momenta $P = \{P_{\xi}\}$. $\mathcal{H}_{int}$ is the linear coupling term of the system and the reservoir. 

For harmonic baths, the reservoir Hamiltonian reads $$\mathcal{H}_{R} = \sum_{\xi} \frac{P_{\xi}^2}{2M} + \frac{1}{2}M \omega_{\xi}^2$$


%\subsection{Quantum Master Equations}

%\subsection{Adiabatic and Markov Approximations}


\section{Path Integral Treatment of Open Quantum Systems}

% Why are path integrals useful for treating quantum systems.
% S + R  => Influence Functionals.


For a general Hamiltonian in cartesian co-ordinates $$\mathcal{H} = \sum_{j} \frac{p_j^2}{2m_j} + V(x_1, ..., x_n)$$

To propagate this system from an initial ($t_0$, $x_0$) to a final ($t_f$, $x_f$), let us split the time scale t into N short time chunks $\Delta t \equiv \frac{t}{N}$. Now the time-evolution operator can be written as :

\begin{equation} \label{4.x}
    e^{-i\mathcal{H}t/\hbar} = e^{-i\mathcal{H}\Delta t/\hbar}...e^{-i\mathcal{H}\Delta t/\hbar} = \prod_{k=1}^{N} e^{-i\mathcal{H}(t_k - t_{k-1})/\hbar}
\end{equation}

where $t_k = kt/N$. 

Using $$\int dx_k \ket{\psi_k}\bra{\psi_k} = 1$$ between successive steps we get :

\begin{equation} \label{4.xx}
    \bra{x_f} e^{-i\mathcal{H}t/\hbar} \ket{x_0} = \int dx_1 ... \int dx_{N-1} \prod_{k=1}^{N} \bra{x_k} e^{-i\mathcal{H}(t_k - t_{k-1})/\hbar} \ket{x_{k-1}}
\end{equation}

This is an exact representation for any value of N. $x_f \equiv x_N$ is the final point in the  propagation.

To evaluate this integral, we need an approximation to evaluate the short time propagator in a time-step $\Delta t$. If operators A and B do not commute, we cannot use the rule of exponential of a sum of operators equals the product of exponentials of the operators. The expansion of $e^{A+B}$ developed by Suzuki and Trotter helps us alleviate this problem. They found that :

\begin{equation}\label{4.x}
e^{\delta(A + B)} = e^{\delta A}e^{\delta B} + O(\delta^2)
\end{equation}

The error term $O(\delta^2)$ vanishes when we take the limit of $\delta \to 0$.

As a guess way to split the Hamiltonian if we separate out the kinetic term $\mathcal{T}$ and the potential term $\mathcal{V}$ in the following way :

\begin{equation}\label{4.x}
e^{-i\mathcal{H}\Delta t/\hbar} \approx e^{-i\mathcal{V}\Delta t/2\hbar}e^{-i\mathcal{T}\Delta t/\hbar}e^{-i\mathcal{V}\Delta t/2\hbar} 
\end{equation}

We can evaluate the kinetic part exactly to get :

\begin{equation} \label{4.x}
\bra{x_k}e^{-i\mathcal{H} \Delta t/\hbar}\ket{x_{k-1}} = \prod_{j=1}^{n} (\frac{m_j}{2\pi i \hbar \Delta t})^{1/2} \exp(\frac{i}{\hbar} \frac{m_j}{2\Delta t} (x_{j,k} - x_{j, k-1})^2)
\end{equation}

Plugging this into the propagation expression we get :

\begin{equation} \label{4.x}
    \begin{split}
        \bra{x_f} e^{-i\mathcal{H}t/\hbar} \ket{x_0} & \approx \prod_{j=1}^{n} (\frac{m_j N}{2\pi i \hbar t})^{1/2}  \int d^n x_1 ... \int d^n x_{N-1} \\ & \exp(\frac{i}{\hbar} \sum_{k=1}^{N} (\sum_{j=1}^{n} \frac{m_j N}{2t} (x_{j,k} - x_{j, k-1})^2 - \frac{t}{2N}[\mathcal{V}(x_k) + \mathcal(V_{k-1})]))
    \end{split}
\end{equation}

This equation is the "primitive" discretized path integral expression. As we use more sophisticated methods to split the Hamiltonian we will get better discretized expressions to work with.

When we take the limit of $N \to \infty$, we get an equality.  

\section{Imaginary Time Path Integral Monte-Carlo}

% Wick rotation, then reduces to canonical ensemble problem, then just sample the integral.

% Also discuss the problems with imaginary time PIMC.


\section{Real Time Path Integral Dynamics}

% Start by discussing the sign problem in real-time PIMC

One of the major issues with performing real-time path integral monte carlo simulations is the infamous sign problem. % Find a good explanation for this.

To alleviate the sign problem, we employ a method developed by Makri and co-workers  to develop equations that converge when attempting to numerically sample the integrals. 

\subsection{Quasi-Adiabatic Propagator Path Integral (QuAPI)}






\subsection{Quantum-Classical Path Integral (QCPI)}

\subsection{Evaluating the Real-Time Path Integrals}

\section{Analytic Continuation Method}

\section{Determining Observables}

\subsection{Mobility}

\subsection{Polaron Radius}

\subsection{Self-Energy}


