% Chapter Template

\chapter{Introduction} % Main chapter title

\label{Chapter1} % Change X to a consecutive number; for referencing this chapter elsewhere, use \ref{ChapterX}

\lhead{Chapter 1. \emph{Introduction}} % Change X to a consecutive number; this is for the header on each page - perhaps a shortened title

\section{Semiconductor Materials}

%What are semiconductors and why are they cool

For a variety of reasons, scientists have taken a great deal of interest in understanding how electrons behave in solids. We can figure out a few structural properties of materials from simply looking at a solid's crystal structure, but if we're interested in harnessing all of the material potential a solid can offer, we need to dive deep into what the electrons are doing. In the early 19th century, some scientists started to take notice of the fact that certain materials were conductive under very specific conditions and in 1874, Karl Braun discovered a junction contact between metal sulphides and pointed metal tips that has the  ability to rectify AC to DC current \cite{lukasiak2010history}. Unwittingly, he had discovered the metal-semiconductor junction!

In the 1930s, after the development of quantum mechanics and quantum statistics, scientists started coming up with better models of solids and started to understand the electrical conductivity of different materials in terms of their band structure.

% Band structure bit more explanation + diagram of MOs forming bands in solids. 

Based on the band structure, scientists were then able to classify materials as conductors, semimetals, semiconductors and insulators.

In 1947, Bardeen, Brittain and Shockley at Bell Labs invented the transistor, a tiny device that could switch electrical signals on and off, or amplify them. This was a breakthrough that lead the foundation for analog and digital electronics. In 1958, Jack Kilby at Texas Instruments unveiled the Integrated Circuit (IC), which was composed of several transistors, resistors and capacitors all combined on a single substrate, paving the way for the modern computing era.

% Re-write the photovoltaics para a bit.

In the 1950s and 1960s, researchers began experimenting with semiconductor materials for converting sunlight into electricity, leading to the development of solar cells. The first practical photovoltaic (PV) cells were made of silicon, a semiconductor material. Over the years, advancements in semiconductor technology, such as the development of multi-junction solar cells and thin-film technologies, have significantly improved the efficiency and affordability of solar panels.In the 1950s and 1960s, researchers began experimenting with semiconductor materials for converting sunlight into electricity, leading to the development of solar cells. The first practical photovoltaic (PV) cells were made of silicon, a semiconductor material. Over the years, advancements in semiconductor technology, such as the development of multi-junction solar cells and thin-film technologies, have significantly improved the efficiency and affordability of solar panels.


% How scientists started using different materials over time for various applications, from Si -> GaAs -> GaN -> Organics.


% For organics, talk about OLEDs a bit (rewrite the OLED para)

Semiconductor materials have also revolutionized the display industry. Liquid Crystal Displays (LCDs), introduced in the 1960s, utilize the properties of liquid crystals, which are organic semiconductor materials. The development of Organic Light-Emitting Diodes (OLEDs) in the 1980s marked another milestone. OLEDs consist of organic semiconductor compounds that emit light when an electric current is applied. OLED displays offer vibrant colors, high contrast ratios, and flexibility, making them ideal for various applications, including smartphones, TVs, and wearable devices.


% Try summarising https://www.feynmanlectures.caltech.edu/III_14.html in its entirety here.

%Electrons in a solid, Kronnig-Penny for demonstrating bands

%Mobility, Einstein-Smoluchowski Relation.

%simple p-n junction because why not.

%Photovoltaics and their importance in modern context of energy materials

%Shockley Queisser limit with proof. Mention multijunction as one way of overcoming but processes in organic materials more interesting.

%Organic photovoltaics are taking off.



\section{Electronic Structure}

The electronic structure is the solution of the quantum states of electrons in a given chemical system. Typically, this involves determining the energies and wavefunctions of the various states. This can be done by solving the Schr{\"o}dinger equation for molecules.

\subsection{Tight-binding Model of Solids}

\subsection{Density Functional Theory}

% DFT and TD-DFT

\subsection {Complete Active Space Methods}

\subsubsection{CASCI}

\subsubsection{CASSCF}

\section{Phonons}

% What are phonons

% Sample phonon dispersion in a solid.

% How to get phonons from electronic structure calculations in solids (phonopy or something of that order).

% Why treating el-ph interactions in solids is important.



